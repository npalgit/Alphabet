%%%%%%%%%%%%%%%%%%%%%%%%%%%%%%%%%%%%%%%%%
% Large Colored Title Article
% LaTeX Template
% Version 1.1 (25/11/12)
%
% This template has been downloaded from:
% http://www.LaTeXTemplates.com
%
% Original author:
% Frits Wenneker (http://www.howtotex.com)
%
% License:
% CC BY-NC-SA 3.0 (http://creativecommons.org/licenses/by-nc-sa/3.0/)
%
%%%%%%%%%%%%%%%%%%%%%%%%%%%%%%%%%%%%%%%%%

%----------------------------------------------------------------------------------------
%	PACKAGES AND OTHER DOCUMENT CONFIGURATIONS
%----------------------------------------------------------------------------------------

\documentclass[DIV=calc, paper=a4, fontsize=11pt, twocolumn]{scrartcl}	 % A4 paper and 11pt font size

\usepackage{lipsum} % Used for inserting dummy 'Lorem ipsum' text into the template
\usepackage[english]{babel} % English language/hyphenation
\usepackage[protrusion=true,expansion=true]{microtype} % Better typography
\usepackage{amsmath,amsfonts,amsthm} % Math packages
\usepackage[svgnames]{xcolor} % Enabling colors by their 'svgnames'
\usepackage[hang, small,labelfont=bf,up,textfont=it,up]{caption} % Custom captions under/above floats in tables or figures
\usepackage{booktabs} % Horizontal rules in tables
\usepackage{fix-cm}	 % Custom font sizes - used for the initial letter in the document

\usepackage{sectsty} % Enables custom section titles
\allsectionsfont{\usefont{OT1}{phv}{b}{n}} % Change the font of all section commands

\usepackage{fancyhdr} % Needed to define custom headers/footers
\pagestyle{fancy} % Enables the custom headers/footers
\usepackage{lastpage} % Used to determine the number of pages in the document (for "Page X of Total")

% Headers - all currently empty
\lhead{}
\chead{}
\rhead{}

% Footers
\lfoot{}
\cfoot{}
\rfoot{\footnotesize Page \thepage\ of \pageref{LastPage}} % "Page 1 of 2"

\renewcommand{\headrulewidth}{0.0pt} % No header rule
\renewcommand{\footrulewidth}{0.4pt} % Thin footer rule

\usepackage{lettrine} % Package to accentuate the first letter of the text
\newcommand{\initial}[1]{ % Defines the command and style for the first letter
\lettrine[lines=3,lhang=0.3,nindent=0em]{
\color{DarkGoldenrod}
{\textsf{#1}}}{}}

%----------------------------------------------------------------------------------------
%	TITLE SECTION
%----------------------------------------------------------------------------------------

\usepackage{titling} % Allows custom title configuration

\newcommand{\HorRule}{\color{DarkGoldenrod} \rule{\linewidth}{1pt}} % Defines the gold horizontal rule around the title

\pretitle{\vspace{-30pt} \begin{flushleft} \HorRule \fontsize{50}{50} \usefont{OT1}{phv}{b}{n} \color{DarkRed} \selectfont} % Horizontal rule before the title

\title{LeetCode Summary} % Your article title

\posttitle{\par\end{flushleft}\vskip 0.5em} % Whitespace under the title

\preauthor{\begin{flushleft}\large \lineskip 0.5em \usefont{OT1}{phv}{b}{sl} \color{DarkRed}} % Author font configuration

\author{Zhaoming Yin, } % Your name

\postauthor{\footnotesize \usefont{OT1}{phv}{m}{sl} \color{Black} % Configuration for the institution name
Oracle Corporation % Your institution

\par\end{flushleft}\HorRule} % Horizontal rule after the title

\date{} % Add a date here if you would like one to appear underneath the title block

%----------------------------------------------------------------------------------------

\begin{document}

\maketitle % Print the title

\thispagestyle{fancy} % Enabling the custom headers/footers for the first page 

%----------------------------------------------------------------------------------------
%	ABSTRACT
%----------------------------------------------------------------------------------------

% The first character should be within \initial{}
\initial{H}\textbf{ere are some Leetcode summaries to help interview in 2016. }

%----------------------------------------------------------------------------------------
%	ARTICLE CONTENTS
%----------------------------------------------------------------------------------------

\section*{Bit Manipulations}

\subsection*{Easy}

\begin{enumerate}
\item Single Number I: \cite{136} 
\item Reverse Bits:    \cite{190}
\item Num of 1 Bits:   \cite{191} 
\item Power of two:    \cite{231} 
\item Missing Number:  \cite{268}  
\end{enumerate}

\subsection*{Median}

\begin{enumerate}
\item Subsets: \cite{078}, II: \cite{090}
\item Maximum Product of Word Lengths: \cite{318} 
\item Generalized Abbreviation: \cite{320}
\item Range bitwise and: \cite{201} 

\item Single Number III: \cite{260} 
\item Single number II: \cite{137}
\item Majority element: \cite{169}
\end{enumerate}

\subsection*{Hard}
\begin{enumerate}
\item Gray code \cite{089} 
\end{enumerate}

%----------------------------------------------------------------------------------------

\section*{Sort}

\subsection*{Easy}

\begin{enumerate}
\item Valid Anagram: \cite{242} 
\item Meeting rooms: \cite{252} 
\item Insert interval: \cite{057}
\item Merge intervals: \cite{056}
\item Meeting rooms II: \cite{253} 
\item Largest number: \cite{179} 
\item H-Index: \cite{274}
\end{enumerate}

\subsection*{Median}

\begin{enumerate}
\item Wiggle sort: \cite{280}
\item Wiggle sort II: \cite{324}
\item Sort List: \cite{148} 
\item Insertion sort list: \cite{147}
\item Best Meeting Point: \cite{296}

\end{enumerate}

\subsection*{Hard}

\begin{enumerate}
\item Maximum Gap \cite{164} 
\end{enumerate}

%----------------------------------------------------------------------------------------

\section*{Two Pointers}

\subsection*{Easy}

\begin{enumerate}
\item Valid Palindrome: \cite{125} 
\item Remove element: \cite{027}
\item Remove duplicate from sorted array: \cite{026}, II \cite{080}
\item Move zeros \cite{283}
\item Two sum \cite{167}, three \cite{015}, four \cite{018}, smaller \cite{259}, Closest \cite{016}
\item Sort colors: \cite{075} 
\item Container with most water: \cite{011}
\item Trapping rain water: \cite{042} 
\end{enumerate}

\subsection*{Median}

\begin{enumerate}
\item Minimum window substring \cite{076}
\item Longest substring with at most two distinct characters \cite{159}
\item Longest substring without repeating characters \cite{003}
\item Minimum size subarray sum \cite{209}
\item {\color{red} Substring with Concatenation of All Words} \cite{030}
\end{enumerate}

\subsection*{Hard}
\begin{enumerate}
\item {\color{red} Find the duplicate number} \cite{287}
\end{enumerate}

%----------------------------------------------------------------------------------------

\section*{Binary Search}

\subsection*{Easy}
\begin{enumerate}
\item First bad version: \cite{278} 
\item Search insert position: \cite{035}
\item Search for a range: \cite{034} 
\item {\color{red} Find peak element}: \cite{162}
\item Search in Rotated Sorted Array: \cite{033}
\item Search in Rotated Sorted Array II (duplicate): \cite{081}
\item {\color{red} Search min in rotated sorted array}: \cite{153}
\item {\color{red} Search min in rotated sorted array II}: \cite{154}
\item H-index II – already sorted: \cite{275}
\end{enumerate}

\subsection*{Median}
\begin{enumerate}
\item Sqrt(x): \cite{069}
\item pow(x,n): \cite{050} 
\item Search a 2D matrix \cite{074}
\item Search a 2D matrix II columns are also sorted \cite{240}
\end{enumerate}

\subsection*{Hard}
\begin{enumerate}
\item {\color{red} divide two integers:} \cite{029}
\item {\color{red} Median of two sorted arrays} \cite{004}
\item {\color{red} Dungeon game}
\item {\color{red} Smallest rectangle enclosing pixels}
\end{enumerate}

%----------------------------------------------------------------------------------------

\section*{BFS}

\subsection*{Easy}
\begin{enumerate}
\item Minimum depth of binary tree \cite{111}
\item Binary tree level order traversal
\item Binary tree level order traversal II
\item Binary tree right side view
\item Binary tree zigzag level order traversal
\item Minimum height trees
\item Number of islands
\item Surrounded regions
\item Walls and gates
\item Shortest distance from all buildings
\item Num of connected components
\item Graph valid tree
\item Course schedule
\item Course schedule II
\end{enumerate}

\subsection*{Median}
\begin{enumerate}
\item Clone graph
\end{enumerate}

\subsection*{Hard}
\begin{enumerate}
\item Word ladder 
\item Word ladder II
\end{enumerate}

%----------------------------------------------------------------------------------------
\bibliography{leet}{}
\bibliographystyle{alpha}

\end{document}
